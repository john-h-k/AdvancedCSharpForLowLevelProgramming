\chapter{Marshaling between C\# and C}
For this chapter, you'll need to create a ChapterFive directory and initialize a Dotnet Console project and to reference ''AdvancedDLSupport'' from nuget.

\section{Struct Layout}
The Layout in Struct is by default set to Sequential and you cannot use Auto layout for marshaling between Managed and Unmanaged code. Explicit Layout allows the you to explicitly define the field offsets in struct layout. This is also what enables you to  create a union in struct.

\lstinputlisting[style=customcs]{codes/Chap5/Chap5Snippet1.cs}

You may have noticed that the Val1 occupied 2 byte slots in the struct rather than Val2 being placed immediately after Val1. This is due to data alignment.  More information on that can be found here: https://software.intel.com/en-us/articles/data-alignment-when-migrating-to-64-bit-intel-architecture

To quote from that link:

\begin{coloredbox}
	The fundamental rule of data alignment is that the safest (and most widely supported) approach relies on what Intel terms "the natural boundaries." Those are the ones that occur when you round up the size of a data item to the next largest size of two, four, eight or 16 bytes. For example, a 10-byte float should be aligned on a 16-byte address, whereas 64-bit integers should be aligned to an eight-byte address. Because this is a 64-bit architecture, pointer sizes are all eight bytes wide, and so they too should align on eight-byte boundaries. - Intel 2018
\end{coloredbox}

The size of the struct shown above is 12 bytes rather than 8 bytes, because the sequential layout rule was being followed. However if you wish to override the behavior on data alignment, you can use Explicit Layout as shown below:
\newpage
\lstinputlisting[style=customcs]{codes/Chap5/Chap5Snippet2.cs}

In this struct, the size would become 8 bytes, because there is no padding required for any dangling member to fits in alignment, so everything fits neatly inside the 8 bytes boundary. Also, in explicit layout, the arrangement of members in struct does not affect how memory is laid out, you define the offsets yourself, so you essentially can arrange the members however you like, though it's recommended to keep member arrangement sequential for readability.

\subsection{Union}
Due to the nature of explicit layout, we can overlap where the field are located in memory by using FieldOffset. If you have for an example have the following code:

\lstinputlisting[style=customcs]{codes/Chap5/Chap5Snippet3.cs}

The size of struct would remain at 1 byte, so no padding is added and both fields are storing into the same memory in the struct.

It is however recommended that you create separate structs for each union definition that you may come across in C native library binding as demonstrated:

\lstinputlisting[style=customcs]{codes/Chap5/Chap5Snippet4.cs}

The resulting size of this struct would be 8 bytes, because it still follows the padding rule for sequential in MyStruct by padding the union struct to fits in the memory boundary.

\subsection{Fixed Size Array}

\subsection{Array of Struct}

\section{Pointer Marshaling}

\section{Function Marshaling}

